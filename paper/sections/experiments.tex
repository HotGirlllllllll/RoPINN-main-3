% !TEX root = ../main.tex
\section{EXPERIMENTAL RESULTS}
\subsection{Research Questions}
We organize experiments around three questions:
\begin{itemize}
  \item \textbf{RQ1:} Does RoPINN-ResFF improve the core reaction benchmark under the same training budget?
  \item \textbf{RQ2:} Is the gain attributable to the backbone change under controlled ablations?
  \item \textbf{RQ3:} Does the method transfer across PDE types beyond reaction?
\end{itemize}

\subsection{Experimental Protocol}
\textbf{Benchmarks.} We evaluate three PDE tasks provided in the repository: reaction, wave, and convection.

\textbf{Metrics.} Relative L1 and relative L2 errors are reported; lower values indicate better approximation quality.

\textbf{Compute and budget.} Main comparisons use CUDA (\texttt{cuda:0}) and a fixed budget of 1000 optimization iterations.

\textbf{Compared settings.} For reaction, we compare: (i) the original backup PINN baseline, (ii) the current PINN in the modified branch, and (iii) RoPINN-ResFF. For wave and convection, we compare PINN versus RoPINN-ResFF.

\textbf{Reproducibility settings.} Multi-seed reports use seed-aligned baseline/ours pairing. Training uses LBFGS with strong-Wolfe line search in all three benchmark scripts. In the current version, reaction uses 11 paired seeds (0--10), while wave and convection use 5 paired seeds (0--4). Table files are generated by \texttt{scripts/paper\_make\_tables.py}, and statistical tests are generated by \texttt{scripts/paper\_significance\_tests.py}.

\subsection{Compute Accounting}
To address fairness concerns beyond fixed-iteration comparison, we report parameter count and end-to-end wall-time under the same iteration budget and device. In the current draft, this profiling is reported on reaction (seed 0) as the main benchmark where statistical gains are established.

\begin{table}[H]
\centering
\caption{Compute accounting across tasks (1000 iterations, same CUDA device). Wall-time is estimated from the final tqdm throughput reported in each corresponding run log.}
\label{tab:compute-accounting}
\begin{tabular}{llccc}
\toprule
Task & Method & Params & Wall-time (s) & Sec/iter \\
\midrule
Reaction & PINN baseline & 527,361 & 317.5 & 0.3175 \\
Reaction & RoPINN-ResFF (ours) & 1,642,497 & 1000.0 & 1.0000 \\
Wave & PINN baseline & 527,361 & 1110.0 & 1.1100 \\
Wave & RoPINN-ResFF (ours) & 1,642,497 & 2170.0 & 2.1700 \\
Convection & PINN baseline & 527,361 & 609.8 & 0.6098 \\
Convection & RoPINN-ResFF (ours) & 1,642,497 & 1120.0 & 1.1200 \\
\bottomrule
\end{tabular}
\end{table}


\subsection{Main Reaction Comparison}
Table~\ref{tab:reaction-main} reports the primary reaction results. The table is auto-generated by \texttt{scripts/paper\_make\_tables.py} from \path{results/paper/summary.csv}.

\begin{table}[H]
\centering
\caption{Main comparison on 1D reaction (1000 iterations, single run).}
\label{tab:reaction-main}
\begin{tabular}{lcc}
\toprule
Method & Relative L1 & Relative L2 \\
\midrule
Original RoPINN PINN baseline (backup) & 0.016627 & 0.030553 \\
Current PINN (modified branch) & 0.043673 & 0.075895 \\
RoPINN-ResFF (ours) & \textbf{0.001947} & \textbf{0.004269} \\
\bottomrule
\end{tabular}
\end{table}


In this single-run comparison, RoPINN-ResFF yields a substantial improvement over the original baseline.

\subsection{Qualitative Visual Comparison}
To complement scalar metrics, Figure~\ref{fig:qualitative-error} compares absolute-error maps between baseline PINN and RoPINN-ResFF on reaction and wave.

\begin{figure}[H]
\centering
\begin{minipage}[t]{0.48\linewidth}
\centering
\includegraphics[width=\linewidth]{../results/1dreaction_PINN_region_optimization_error_paper_curr_pinn.pdf}\\
\footnotesize Reaction: PINN
\end{minipage}
\hfill
\begin{minipage}[t]{0.48\linewidth}
\centering
\includegraphics[width=\linewidth]{../results/1dreaction_PINN_ResFF_region_optimization_error_paper_resff_only.pdf}\\
\footnotesize Reaction: RoPINN-ResFF
\end{minipage}

\vspace{0.6em}

\begin{minipage}[t]{0.48\linewidth}
\centering
\includegraphics[width=\linewidth]{../results/1dwave_PINN_region_optimization_error_paper_wave_pinn.pdf}\\
\footnotesize Wave: PINN
\end{minipage}
\hfill
\begin{minipage}[t]{0.48\linewidth}
\centering
\includegraphics[width=\linewidth]{../results/1dwave_PINN_ResFF_region_optimization_error_paper_wave_resff.pdf}\\
\footnotesize Wave: RoPINN-ResFF
\end{minipage}

\caption{Task-wise qualitative error comparison on reaction and wave. Lower-intensity error regions are more prominent for RoPINN-ResFF. Convection behavior is quantified in Tables~\ref{tab:conv-mseed} and \ref{tab:stat-tests}.}
\label{fig:qualitative-error}
\end{figure}

\subsection{Reaction Multi-Seed Robustness}
To address stochastic variance, we evaluate reaction with 11 random seeds for baseline and RoPINN-ResFF.

\begin{table}[H]
\centering
\caption{Reaction benchmark with 5 seeds (1000 iterations).}
\label{tab:reaction-mseed}
\begin{tabular}{lcc}
\toprule
Method & Relative L1 (mean $\pm$ std) & Relative L2 (mean $\pm$ std) \\
\midrule
Baseline PINN & 0.217351 $\pm$ 0.380037 & 0.233924 $\pm$ 0.371040 \\
RoPINN-ResFF (ours) & \textbf{0.002978 $\pm$ 0.000653} & \textbf{0.007432 $\pm$ 0.002161} \\
\bottomrule
\end{tabular}
\end{table}


Compared with baseline mean errors, RoPINN-ResFF achieves substantially lower error and lower variance. The baseline also exhibits occasional failure cases, whereas RoPINN-ResFF remains stable under seed variation. Relative to baseline means, RoPINN-ResFF reduces L1 by 98.38\% and L2 by 96.21\%.

\subsection{Budget Sensitivity on Reaction}
To address concerns that 1000 iterations may be too short for meaningful comparison, we further evaluate reaction under larger budgets (3000 and 5000 iterations) with paired seeds 0--2.

\begin{table}[H]
\centering
\caption{Reaction budget-sensitivity results (paired seeds 0--2).}
\label{tab:reaction-budget}
\begin{tabular}{lcc}
\toprule
Setting & Relative L1 (mean $\pm$ std) & Relative L2 (mean $\pm$ std) \\
\midrule
Baseline, 3000 iters & 0.025193 $\pm$ 0.006149 & 0.045810 $\pm$ 0.010922 \\
RoPINN-ResFF, 3000 iters & \textbf{0.002890 $\pm$ 0.000747} & \textbf{0.007202 $\pm$ 0.002422} \\
Baseline, 5000 iters & 0.025262 $\pm$ 0.006234 & 0.045929 $\pm$ 0.011059 \\
RoPINN-ResFF, 5000 iters & \textbf{0.002871 $\pm$ 0.000746} & \textbf{0.007145 $\pm$ 0.002424} \\
\bottomrule
\end{tabular}
\end{table}



The advantage of RoPINN-ResFF remains stable at higher budgets: compared with baseline, L1/L2 reductions are 88.53\%/84.28\% at 3000 iterations and 88.64\%/84.44\% at 5000 iterations. Moreover, the baseline and RoPINN-ResFF means change only marginally from 3000 to 5000 iterations, indicating near-plateau behavior in this setting rather than a purely early-iteration artifact.

\subsection{Wave Multi-Seed Generalization}
\begin{table}[H]
\centering
\caption{Wave benchmark with 5 seeds (1000 iterations).}
\label{tab:wave-mseed}
\begin{tabular}{lcc}
\toprule
Method & Relative L1 (mean $\pm$ std) & Relative L2 (mean $\pm$ std) \\
\midrule
Baseline PINN & 0.060335 $\pm$ 0.060530 & 0.062967 $\pm$ 0.065164 \\
RoPINN-ResFF (ours) & \textbf{0.011916 $\pm$ 0.001465} & \textbf{0.012183 $\pm$ 0.001683} \\
\bottomrule
\end{tabular}
\end{table}


In multi-seed statistics, RoPINN-ResFF improves wave mean errors and exhibits lower variance than baseline.

\subsection{Convection Multi-Seed Generalization}
\begin{table}[H]
\centering
\caption{Convection benchmark with 5 seeds (1000 iterations).}
\label{tab:conv-mseed}
\begin{tabular}{lcc}
\toprule
Method & Relative L1 (mean $\pm$ std) & Relative L2 (mean $\pm$ std) \\
\midrule
Baseline PINN & \textbf{0.619487 $\pm$ 0.052273} & \textbf{0.699839 $\pm$ 0.048869} \\
RoPINN-ResFF (ours) & 1.015963 $\pm$ 0.021507 & 1.039029 $\pm$ 0.040077 \\
\bottomrule
\end{tabular}
\end{table}


For convection, RoPINN-ResFF yields higher mean errors than baseline, indicating a clear task mismatch under the current architecture and hyperparameter configuration.

\subsection{Statistical Tests and Confidence Intervals}
To supplement mean$\pm$std reporting, we conduct paired significance analysis on seed-aligned runs (same seed IDs in baseline and ours). Specifically, we report exact two-sided paired randomization tests and paired bootstrap 95\% confidence intervals for
\[
\Delta=\mathrm{mean}(\mathrm{ours}-\mathrm{baseline}),
\]
where negative $\Delta$ indicates improvement.

\begin{table}[H]
\centering
\caption{Statistical test summary across paired multi-seed runs. $\Delta=\mathrm{mean}(\mathrm{ours}-\mathrm{baseline})$; negative values indicate improvement. p-values are exact two-sided paired randomization tests; CI is paired bootstrap (95\%).}
\label{tab:stat-tests}
\begin{tabular}{llccc}
\toprule
Task & Metric & $\Delta$ (ours-baseline) & 95\% CI of $\Delta$ & p-value \\
\midrule
Reaction & L1 & -0.201094 & [-0.457322, -0.027849] & <0.001 \\
Reaction & L2 & -0.215465 & [-0.462917, -0.046427] & <0.001 \\
Wave & L1 & -0.048419 & [-0.108907, -0.012743] & 0.0625 \\
Wave & L2 & -0.050784 & [-0.115674, -0.012861] & 0.0625 \\
Convection & L1 & 0.396476 & [0.335960, 0.457900] & 0.0625 \\
Convection & L2 & 0.339190 & [0.271985, 0.418464] & 0.0625 \\
\bottomrule
\end{tabular}
\end{table}


Across tasks and metrics, confidence intervals characterize both effect direction and effect magnitude. For reaction (11 paired seeds), both metrics are statistically significant under exact paired randomization tests (p=0.00098 for L1 and L2). Wave and convection currently use smaller seed counts; therefore, we keep claims task-specific and avoid over-generalization.

\subsection{Ablation Insight}
For reaction, the primary gain is attributable to the backbone upgrade (ResFF), with consistent error reductions under both single-run and multi-seed reporting.

\subsection{Reproducibility Artifacts}
The repository already stores paper-ready artifacts:
\begin{itemize}
  \item summary tables/plots under \path{results/paper/};
  \item seed-level data and significance results under \path{results/paper/tables/};
  \item run-level metric CSV files for each run tag;
  \item prediction/error/loss PDFs for each benchmark and run tag.
\end{itemize}

These files provide direct traceability from reported numbers to raw outputs.

\subsection{Validity Scope}
Multi-seed evidence is available for all three tasks, but seed counts are asymmetric in the current draft (reaction: 11, wave/convection: 5). The method remains clearly task-dependent: it is strong on reaction and wave, but consistently weaker on convection under the current design.
