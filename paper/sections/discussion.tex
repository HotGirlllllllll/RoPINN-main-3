% !TEX root = ../main.tex
\section{DISCUSSION}
\subsection{Why Convection Degrades}
Convection-dominated transport is sensitive to directional structure and often exhibits sharper fronts than reaction or wave settings. The current RoPINN-ResFF design uses isotropic Fourier-feature mapping and generic residual blocks, which may emphasize global frequency fitting while underutilizing advection-specific inductive bias. This mismatch likely contributes to the consistently worse convection errors.

\subsection{Practical Takeaway}
The method should be presented as a \textbf{task-conditional enhancement} of RoPINN:
\begin{itemize}
  \item statistically significant gains on reaction;
  \item consistent directional gains on wave;
  \item clear degradation on convection under the current configuration.
\end{itemize}
This positioning is scientifically stronger than claiming universal superiority. In addition, reaction budget-sensitivity runs (3000/5000 iterations) indicate that gains persist beyond the 1000-iteration setting.

\subsection{On Asymmetric Evaluation}
Given finite compute resources, we prioritize reaction as the confirmatory task (11 seeds with formal significance tests) and treat wave/convection as transfer checks (5 seeds, directional interpretation). Compute accounting is reported for all three tasks under the same 1000-iteration protocol, but confirmatory statistical claims remain restricted to reaction.

\subsection{Capacity vs. Architecture}
On reaction, a direct capacity-matched control further clarifies the source of improvement. Under nearly identical parameter counts (PINN: 1,643,481 vs. RoPINN-ResFF: 1,642,497), RoPINN-ResFF still substantially outperforms the matched PINN. Therefore, the observed gain cannot be attributed only to parameter expansion and is consistent with a genuine architecture effect.

\subsection{Limitations and Future Work}
The current empirical scope remains limited in three aspects. First, seed counts are unbalanced across tasks (reaction uses more paired seeds than wave/convection), which limits cross-task statistical symmetry. Second, convection underperformance is currently characterized but not yet fully resolved by a validated task-specific inductive bias. Third, compute accounting is currently estimated from final progress-bar throughput in single-device runs; a stronger systems-level study should add repeated wall-time profiling under controlled GPU isolation. Addressing these points is part of our ongoing work.
